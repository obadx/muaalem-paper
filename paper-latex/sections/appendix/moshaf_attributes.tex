\subsection{Moshaf Attribute Definitions}
\label{sec:moshaf_attributes}
\begin{itemize}
\item \textbf{rewaya} (\arb{الرواية})
  \begin{itemize}
  \item Values: - \texttt{hafs} (\arb{حفص})
  \item Default Value: 
  \item More Info: The type of the quran Rewaya.
  \end{itemize}

\item \textbf{recitation\_speed} (\arb{سرعة التلاوة})
  \begin{itemize}
  \item Values: 
    \begin{itemize}
    \item  \texttt{mujawad} (\arb{مجود})
    \item  \texttt{above\_murattal} (\arb{فويق المرتل})
    \item  \texttt{murattal} (\arb{مرتل})
    \item  \texttt{hadr} (\arb{حدر})
    \end{itemize}
  \item Default Value: \texttt{murattal} (\arb{مرتل})
  \item More Info: The recitation speed sorted from slowest to the fastest \arb{سرعة التلاوة مرتبة من الأبطأ إلي الأسرع}
  \end{itemize}

\item \textbf{takbeer} (\arb{التكبير})
  \begin{itemize}
  \item Values: 
    \begin{itemize}
    \item  \texttt{no\_takbeer} (\arb{لا تكبير})
    \item  \texttt{beginning\_of\_sharh} (\arb{التكبير من أول الشرح لأول الناس})
    \item  \texttt{end\_of\_doha} (\arb{التكبير من آخر الضحى لآخر الناس})
    \item  \texttt{general\_takbeer} (\arb{التكبير أول كل سورة إلا التوبة})
    \end{itemize}
  \item Default Value: \texttt{no\_takbeer} (\arb{لا تكبير})
  \item More Info: The ways to add takbeer (\arb{الله أكبر}) after Istiaatha (\arb{استعاذة}) and between end of the surah and beginning of the surah. \texttt{no\_takbeer}: "\arb{لا تكبير}" — No Takbeer (No proclamation of greatness, i.e., there is no Takbeer recitation) \texttt{beginning\_of\_sharh}: "\arb{التكبير من أول الشرح لأول الناس}" — Takbeer from the beginning of Surah Ash-Sharh to the beginning of Surah An-Nas \texttt{end\_of\_dohaf}: "\arb{التكبير من آخر الضحى لآخر الناس}" — Takbeer from the end of Surah Ad-Duha to the end of Surah An-Nas \texttt{general\_takbeer}: "\arb{التكبير أول كل سورة إلا التوبة}" — Takbeer at the beginning of every Surah except Surah At-Tawbah
  \end{itemize}

\item \textbf{madd\_monfasel\_len} (\arb{مد المنفصل})
  \begin{itemize}
  \item Values: 
    \begin{itemize}
    \item  \texttt{2}
    \item  \texttt{3}
    \item  \texttt{4}
    \item  \texttt{5}
    \end{itemize}
  \item Default Value: 
  \item More Info: The length of Mad Al Monfasel "\arb{مد المنفصل}" for Hafs Rewaya.
  \end{itemize}

\item \textbf{madd\_mottasel\_len} (\arb{مقدار المد المتصل})
  \begin{itemize}
  \item Values: 
    \begin{itemize}
    \item  \texttt{4}
    \item  \texttt{5}
    \item  \texttt{6}
    \end{itemize}
  \item Default Value: 
  \item More Info: The length of Mad Al Motasel "\arb{مد المتصل}" for Hafs.
  \end{itemize}

\item \textbf{madd\_mottasel\_waqf} (\arb{مقدار المد المتصل وقفا})
  \begin{itemize}
  \item Values: 
    \begin{itemize}
    \item  \texttt{4}
    \item  \texttt{5}
    \item  \texttt{6}
    \end{itemize}
  \item Default Value: 
  \item More Info: The length of Madd Almotasel at pause for Hafs.. Example "\arb{السماء}".
  \end{itemize}

\item \textbf{madd\_aared\_len} (\arb{مقدار المد العارض})
  \begin{itemize}
  \item Values: 
    \begin{itemize}
    \item  \texttt{2}
    \item  \texttt{4}
    \item  \texttt{6}
    \end{itemize}
  \item Default Value: 
  \item More Info: The length of Mad Al Aared "\arb{مد العارض للسكون}".
  \end{itemize}

\item \textbf{madd\_alleen\_len} (\arb{مقدار مد اللين})
  \begin{itemize}
  \item Values: 
    \begin{itemize}
    \item  \texttt{2}
    \item  \texttt{4}
    \item  \texttt{6}
    \end{itemize}
  \item Default Value: \texttt{None}
  \item More Info: The length of the Madd al-Leen when stopping at the end of a word (for a sakin waw or ya preceded by a letter with a fatha) should be less than or equal to the length of Madd al-'Arid (the temporary stretch due to stopping). \textbf{Default Value is equal to \texttt{madd\_aared\_len}}. \arb{مقدار مد اللين عن القوف (للواو الساكنة والياء الساكنة وقبلها حرف مفتوح) ويجب أن يكون مقدار مد اللين أقل من أو يساوي مع العارض}
  \end{itemize}

\item \textbf{ghonna\_lam\_and\_raa} (\arb{غنة اللام و الراء})
  \begin{itemize}
  \item Values: 
    \begin{itemize}
    \item  \texttt{ghonna} (\arb{غنة})
    \item  \texttt{no\_ghonna} (\arb{لا غنة})
    \end{itemize}
  \item Default Value: \texttt{no\_ghonna} (\arb{لا غنة})
  \item More Info: The ghonna for merging (Idghaam) noon with Lam and Raa for Hafs.
  \end{itemize}

\item \textbf{meem\_aal\_imran} (\arb{ميم آل عمران في قوله تعالى: \{الم الله\} وصلا})
  \begin{itemize}
  \item Values: 
    \begin{itemize}
    \item  \texttt{waqf} (\arb{وقف})
    \item  \texttt{wasl\_2} (\arb{فتح الميم ومدها حركتين})
    \item  \texttt{wasl\_6} (\arb{فتح الميم ومدها ستة حركات})
    \end{itemize}
  \item Default Value: \texttt{waqf} (\arb{وقف})
  \item More Info: The ways to recite the word meem Aal Imran (\arb{الم الله}) at connected recitation. \texttt{waqf}: Pause with a prolonged madd (elongation) of 6 harakat (beats). \texttt{wasl\_2} Pronounce "meem" with fathah (a short "a" sound) and stretch it for 2 harakat. \texttt{wasl\_6} Pronounce "meem" with fathah and stretch it for 6 harakat.
  \end{itemize}

\item \textbf{madd\_yaa\_alayn\_alharfy} (\arb{مقدار   المد اللازم الحرفي للعين})
  \begin{itemize}
  \item Values: 
    \begin{itemize}
    \item  \texttt{2}
    \item  \texttt{4}
    \item  \texttt{6}
    \end{itemize}
  \item Default Value: \texttt{6}
  \item More Info: The length of Lzem Harfy of Yaa in letter Al-Ayen Madd "\arb{المد الحرفي اللازم لحرف العين}" in surar: Maryam "\arb{مريم}", AlShura "\arb{الشورى}".
  \end{itemize}

\item \textbf{saken\_before\_hamz} (\arb{الساكن قبل الهمز})
  \begin{itemize}
  \item Values: 
    \begin{itemize}
    \item  \texttt{tahqeek} (\arb{تحقيق})
    \item  \texttt{general\_sakt} (\arb{سكت عام})
    \item  \texttt{local\_sakt} (\arb{سكت خاص})
    \end{itemize}
  \item Default Value: \texttt{tahqeek} (\arb{تحقيق})
  \item More Info: The ways of Hafs for saken before hamz. "The letter with sukoon before the hamzah (\arb{ء})".And it has three forms: full articulation (\texttt{tahqeeq}), general pause (\texttt{general\_sakt}), and specific pause (\texttt{local\_skat}).
  \end{itemize}

\item \textbf{sakt\_iwaja} (\arb{السكت عند عوجا في الكهف})
  \begin{itemize}
  \item Values: 
    \begin{itemize}
    \item  \texttt{sakt} (\arb{سكت})
    \item  \texttt{waqf} (\arb{وقف})
    \item  \texttt{idraj} (\arb{إدراج})
    \end{itemize}
  \item Default Value: \texttt{waqf} (\arb{وقف})
  \item More Info: The ways to recite the word "\arb{عوجا}" (Iwaja). \texttt{sakt} means slight pause. \texttt{idraj} means not \texttt{sakt}. \texttt{waqf}: means full pause, so we can not determine whether the reciter uses \texttt{sakt} or \texttt{idraj} (no sakt).
  \end{itemize}

\item \textbf{sakt\_marqdena} (\arb{السكت عند مرقدنا  في يس})
  \begin{itemize}
  \item Values: 
    \begin{itemize}
    \item  \texttt{sakt} (\arb{سكت})
    \item  \texttt{waqf} (\arb{وقف})
    \item  \texttt{idraj} (\arb{إدراج})
    \end{itemize}
  \item Default Value: \texttt{waqf} (\arb{وقف})
  \item More Info: The ways to recite the word "\arb{مرقدنا}" (Marqadena) in Surat Yassen. \texttt{sakt} means slight pause. \texttt{idraj} means not \texttt{sakt}. \texttt{waqf}: means full pause, so we can not determine whether the reciter uses \texttt{sakt} or \texttt{idraj} (no sakt).
  \end{itemize}

\item \textbf{sakt\_man\_raq} (\arb{السكت عند  من راق في القيامة})
  \begin{itemize}
  \item Values: 
    \begin{itemize}
    \item  \texttt{sakt} (\arb{سكت})
    \item  \texttt{waqf} (\arb{وقف})
    \item  \texttt{idraj} (\arb{إدراج})
    \end{itemize}
  \item Default Value: \texttt{sakt} (\arb{سكت})
  \item More Info: The ways to recite the word "\arb{من راق}" (Man Raq) in Surat Al Qiyama. \texttt{sakt} means slight pause. \texttt{idraj} means not \texttt{sakt}. \texttt{waqf}: means full pause, so we can not determine whether the reciter uses \texttt{sakt} or \texttt{idraj} (no sakt).
  \end{itemize}

\item \textbf{sakt\_bal\_ran} (\arb{السكت عند  بل ران في  المطففين})
  \begin{itemize}
  \item Values: 
    \begin{itemize}
    \item  \texttt{sakt} (\arb{سكت})
    \item  \texttt{waqf} (\arb{وقف})
    \item  \texttt{idraj} (\arb{إدراج})
    \end{itemize}
  \item Default Value: \texttt{sakt} (\arb{سكت})
  \item More Info: The ways to recite the word "\arb{بل ران}" (Bal Ran) in Surat Al Motaffin. \texttt{sakt} means slight pause. \texttt{idraj} means not \texttt{sakt}. \texttt{waqf}: means full pause, so we can not determine whether the reciter uses \texttt{sakt} or \texttt{idraj} (no sakt).
  \end{itemize}

\item \textbf{sakt\_maleeyah} (\arb{وجه  قوله تعالى \{ماليه هلك\} بالحاقة})
  \begin{itemize}
  \item Values: 
    \begin{itemize}
    \item  \texttt{sakt} (\arb{سكت})
    \item  \texttt{waqf} (\arb{وقف})
    \item  \texttt{idgham} (\arb{إدغام})
    \end{itemize}
  \item Default Value: \texttt{waqf} (\arb{وقف})
  \item More Info: The ways to recite the word \{\arb{ماليه هلك}\} in Surah Al-Ahqaf. \texttt{sakt} means slight pause. \texttt{idgham} Assimilation of the letter 'Ha' (\arb{ه}) into the letter 'Ha' (\arb{ه}) with complete assimilation.\texttt{waqf}: means full pause, so we can not determine whether the reciter uses \texttt{sakt} or \texttt{idgham}.
  \end{itemize}

\item \textbf{between\_anfal\_and\_tawba} (\arb{وجه بين الأنفال والتوبة})
  \begin{itemize}
  \item Values: 
    \begin{itemize}
    \item  \texttt{waqf} (\arb{وقف})
    \item  \texttt{sakt} (\arb{سكت})
    \item  \texttt{wasl} (\arb{وصل})
    \end{itemize}
  \item Default Value: \texttt{waqf} (\arb{وقف})
  \item More Info: The ways to recite end of Surah Al-Anfal and beginning of Surah At-Tawbah.
  \end{itemize}

\item \textbf{noon\_and\_yaseen} (\arb{الإدغام والإظهار في النون عند الواو من قوله تعالى: \{يس والقرآن\}و \{ن والقلم\}})
  \begin{itemize}
  \item Values: 
    \begin{itemize}
    \item  \texttt{izhar} (\arb{إظهار})
    \item  \texttt{idgham} (\arb{إدغام})
    \end{itemize}
  \item Default Value: \texttt{izhar} (\arb{إظهار})
  \item More Info: Whether to merge noon of both: \{\arb{يس}\} and \{\arb{ن}\} with (\arb{و}) "\texttt{idgham}" or not "\texttt{izhar}".
  \end{itemize}

\item \textbf{yaa\_ataan} (\arb{إثبات الياء وحذفها وقفا في قوله تعالى \{آتان\} بالنمل})
  \begin{itemize}
  \item Values: 
    \begin{itemize}
    \item  \texttt{wasl} (\arb{وصل})
    \item  \texttt{hadhf} (\arb{حذف})
    \item  \texttt{ithbat} (\arb{إثبات})
    \end{itemize}
  \item Default Value: \texttt{wasl} (\arb{وصل})
  \item More Info: The affirmation and omission of the letter 'Yaa' in the pause of the verse \{\arb{آتاني}\} in Surah An-Naml. \texttt{wasl}: means connected recitation without pausing as (\arb{آتانيَ}). \texttt{hadhf}: means deletion of letter (\arb{ي}) at pause so recited as (\arb{آتان}). \texttt{ithbat}: means confirmation reciting letter (\arb{ي}) at pause as (\arb{آتاني}).
  \end{itemize}

\item \textbf{start\_with\_ism} (\arb{وجه البدأ بكلمة \{الاسم\} في سورة الحجرات})
  \begin{itemize}
  \item Values: 
    \begin{itemize}
    \item  \texttt{wasl} (\arb{وصل})
    \item  \texttt{lism} (\arb{لسم})
    \item  \texttt{alism} (\arb{ألسم})
    \end{itemize}
  \item Default Value: \texttt{wasl} (\arb{وصل})
  \item More Info: The ruling on starting with the word \{\arb{الاسم}\} in Surah Al-Hujurat. \texttt{lism} Recited as (\arb{لسم}) at the beginning. \texttt{alism} Recited as (\arb{ألسم}). \texttt{wasl}: means completing recitation without pausing as normal, So Reciting is as (\arb{بئس لسم}).
  \end{itemize}

\item \textbf{yabsut} (\arb{السين والصاد في قوله تعالى: \{والله يقبض ويبسط\} بالبقرة})
  \begin{itemize}
  \item Values: 
    \begin{itemize}
    \item  \texttt{seen} (\arb{سين})
    \item  \texttt{saad} (\arb{صاد})
    \end{itemize}
  \item Default Value: \texttt{seen} (\arb{سين})
  \item More Info: The ruling on pronouncing \texttt{seen} (\arb{س}) or \texttt{saad} (\arb{ص}) in the verse \{\arb{والله يقبض ويبسط}\} in Surah Al-Baqarah.
  \end{itemize}

\item \textbf{bastah} (\arb{السين والصاد في قوله تعالى:  \{وزادكم في الخلق بسطة\} بالأعراف})
  \begin{itemize}
  \item Values: 
    \begin{itemize}
    \item  \texttt{seen} (\arb{سين})
    \item  \texttt{saad} (\arb{صاد})
    \end{itemize}
  \item Default Value: \texttt{seen} (\arb{سين})
  \item More Info: The ruling on pronouncing \texttt{seen} (\arb{س}) or \texttt{saad} (\arb{ص}) in the verse \{\arb{وزادكم في الخلق بسطة}\} in Surah Al-A'raf.
  \end{itemize}

\item \textbf{almusaytirun} (\arb{السين والصاد في قوله تعالى \{أم هم المصيطرون\} بالطور})
  \begin{itemize}
  \item Values: 
    \begin{itemize}
    \item  \texttt{seen} (\arb{سين})
    \item  \texttt{saad} (\arb{صاد})
    \end{itemize}
  \item Default Value: \texttt{saad} (\arb{صاد})
  \item More Info: The pronunciation of \texttt{seen} (\arb{س}) or \texttt{saad} (\arb{ص}) in the verse \{\arb{أم هم المصيطرون}\} in Surah At-Tur.
  \end{itemize}

\item \textbf{bimusaytir} (\arb{السين والصاد في قوله تعالى:  \{لست عليهم بمصيطر\} بالغاشية})
  \begin{itemize}
  \item Values: 
    \begin{itemize}
    \item  \texttt{seen} (\arb{سين})
    \item  \texttt{saad} (\arb{صاد})
    \end{itemize}
  \item Default Value: \texttt{saad} (\arb{صاد})
  \item More Info: The pronunciation of \texttt{seen} (\arb{س}) or \texttt{saad} (\arb{ص}) in the verse \{\arb{لست عليهم بمصيطر}\} in Surah Al-Ghashiyah.
  \end{itemize}

\item \textbf{tasheel\_or\_madd} (\arb{همزة الوصل في قوله تعالى: \{آلذكرين\} بموضعي الأنعام و\{آلآن\} موضعي يونس و\{آلله\} بيونس والنمل})
  \begin{itemize}
  \item Values: 
    \begin{itemize}
    \item  \texttt{tasheel} (\arb{تسهيل})
    \item  \texttt{madd} (\arb{مد})
    \end{itemize}
  \item Default Value: \texttt{madd} (\arb{مد})
  \item More Info: Tasheel of Madd "\arb{وجع التسهيل أو المد}" for 6 words in The Holy Quran: "\arb{ءالذكرين}", "\arb{ءالله}", "\arb{ءائن}".
  \end{itemize}

\item \textbf{yalhath\_dhalik} (\arb{الإدغام وعدمه في قوله تعالى: \{يلهث ذلك\} بالأعراف})
  \begin{itemize}
  \item Values: 
    \begin{itemize}
    \item  \texttt{izhar} (\arb{إظهار})
    \item  \texttt{idgham} (\arb{إدغام})
    \item  \texttt{waqf} (\arb{وقف})
    \end{itemize}
  \item Default Value: \texttt{idgham} (\arb{إدغام})
  \item More Info: The assimilation (\texttt{idgham}) and non-assimilation (\texttt{izhar}) in the verse \{\arb{يلهث ذلك}\} in Surah Al-A'raf. \texttt{waqf}: means the reciter has paused on (\arb{يلهث})
  \end{itemize}

\item \textbf{irkab\_maana} (\arb{الإدغام والإظهار في قوله تعالى: \{اركب معنا\} بهود})
  \begin{itemize}
  \item Values: 
    \begin{itemize}
    \item  \texttt{izhar} (\arb{إظهار})
    \item  \texttt{idgham} (\arb{إدغام})
    \item  \texttt{waqf} (\arb{وقف})
    \end{itemize}
  \item Default Value: \texttt{idgham} (\arb{إدغام})
  \item More Info: The assimilation and clear pronunciation in the verse \{\arb{اركب معنا}\} in Surah Hud. This refers to the recitation rules concerning whether the letter "Noon" (\arb{ن}) is assimilated into the following letter or pronounced clearly when reciting this specific verse. \texttt{waqf}: means the reciter has paused on (\arb{اركب})
  \end{itemize}

\item \textbf{noon\_tamnna} (\arb{الإشمام والروم (الاختلاس) في قوله تعالى \{لا تأمنا على يوسف\}})
  \begin{itemize}
  \item Values: 
    \begin{itemize}
    \item  \texttt{ishmam} (\arb{إشمام})
    \item  \texttt{rawm} (\arb{روم})
    \end{itemize}
  \item Default Value: \texttt{ishmam} (\arb{إشمام})
  \item More Info: The nasalization (\texttt{ishmam}) or the slight drawing (\texttt{rawm}) in the verse \{\arb{لا تأمنا على يوسف}\}
  \end{itemize}

\item \textbf{harakat\_daaf} (\arb{حركة الضاد (فتح أو ضم) في قوله تعالى \{ضعف\} بالروم})
  \begin{itemize}
  \item Values: 
    \begin{itemize}
    \item  \texttt{fath} (\arb{فتح})
    \item  \texttt{dam} (\arb{ضم})
    \end{itemize}
  \item Default Value: \texttt{fath} (\arb{فتح})
  \item More Info: The vowel movement of the letter 'Dhad' (\arb{ض}) (whether with \texttt{fath} or \texttt{dam}) in the word \{\arb{ضعف}\} in Surah Ar-Rum.
  \end{itemize}

\item \textbf{alif\_salasila} (\arb{إثبات الألف وحذفها وقفا في قوله تعالى: \{سلاسلا\} بسورة الإنسان})
  \begin{itemize}
  \item Values: 
    \begin{itemize}
    \item  \texttt{hadhf} (\arb{حذف})
    \item  \texttt{ithbat} (\arb{إثبات})
    \item  \texttt{wasl} (\arb{وصل})
    \end{itemize}
  \item Default Value: \texttt{wasl} (\arb{وصل})
  \item More Info: Affirmation and omission of the 'Alif' when pausing in the verse \{\arb{سلاسلا}\} in Surah Al-Insan. This refers to the recitation rule regarding whether the final "Alif" in the word "\arb{سلاسلا}" is pronounced (affirmed) or omitted when pausing (waqf) at this word during recitation in the specific verse from Surah Al-Insan. \texttt{hadhf}: means to remove alif (\arb{ا}) during pause as (\arb{سلاسل}) \texttt{ithbat}: means to recite alif (\arb{ا}) during pause as (\arb{سلاسلا}) \texttt{wasl} means completing the recitation as normal without pausing, so recite it as (\arb{سلاسلَ وأغلالا})
  \end{itemize}

\item \textbf{idgham\_nakhluqkum} (\arb{إدغام القاف في الكاف إدغاما ناقصا أو كاملا \{نخلقكم\} بالمرسلات})
  \begin{itemize}
  \item Values: 
    \begin{itemize}
    \item  \texttt{idgham\_kamil} (\arb{إدغام كامل})
    \item  \texttt{idgham\_naqis} (\arb{إدغام ناقص})
    \end{itemize}
  \item Default Value: \texttt{idgham\_kamil} (\arb{إدغام كامل})
  \item More Info: Assimilation of the letter 'Qaf' into the letter 'Kaf,' whether incomplete (\texttt{idgham\_naqis}) or complete (\texttt{idgham\_kamil}), in the verse \{\arb{نخلقكم}\} in Surah Al-Mursalat.
  \end{itemize}

\item \textbf{raa\_firq} (\arb{التفخيم والترقيق في راء \{فرق\} في الشعراء وصلا})
  \begin{itemize}
  \item Values: 
    \begin{itemize}
    \item  \texttt{waqf} (\arb{وقف})
    \item  \texttt{tafkheem} (\arb{تفخيم})
    \item  \texttt{tarqeeq} (\arb{ترقيق})
    \end{itemize}
  \item Default Value: \texttt{tafkheem} (\arb{تفخيم})
  \item More Info: Emphasis and softening of the letter 'Ra' in the word \{\arb{فرق}\} in Surah Ash-Shu'ara' when connected (wasl). This refers to the recitation rules concerning whether the letter "Ra" (\arb{ر}) in the word "\arb{فرق}" is pronounced with emphasis (\texttt{tafkheem}) or softening (\texttt{tarqeeq}) when reciting the specific verse from Surah Ash-Shu'ara' in connected speech. \texttt{waqf}: means pausing so we only have one way (tafkheem of Raa)
  \end{itemize}

\item \textbf{raa\_alqitr} (\arb{التفخيم والترقيق في راء \{القطر\} في سبأ وقفا})
  \begin{itemize}
  \item Values: 
    \begin{itemize}
    \item  \texttt{wasl} (\arb{وصل})
    \item  \texttt{tafkheem} (\arb{تفخيم})
    \item  \texttt{tarqeeq} (\arb{ترقيق})
    \end{itemize}
  \item Default Value: \texttt{wasl} (\arb{وصل})
  \item More Info: Emphasis and softening of the letter 'Ra' in the word \{\arb{القطر}\} in Surah Saba' when pausing (waqf). This refers to the recitation rules regarding whether the letter "Ra" (\arb{ر}) in the word "\arb{القطر}" is pronounced with emphasis (\texttt{tafkheem}) or softening (\texttt{tarqeeq}) when pausing at this word in Surah Saba'. \texttt{wasl}: means not pausing so we only have one way (tarqeeq of Raa)
  \end{itemize}

\item \textbf{raa\_misr} (\arb{التفخيم والترقيق في راء \{مصر\} في يونس وموضعي يوسف والزخرف  وقفا})
  \begin{itemize}
  \item Values: 
    \begin{itemize}
    \item  \texttt{wasl} (\arb{وصل})
    \item  \texttt{tafkheem} (\arb{تفخيم})
    \item  \texttt{tarqeeq} (\arb{ترقيق})
    \end{itemize}
  \item Default Value: \texttt{wasl} (\arb{وصل})
  \item More Info: Emphasis and softening of the letter 'Ra' in the word \{\arb{مصر}\} in Surah Yunus, and in the locations of Surah Yusuf and Surah Az-Zukhruf when pausing (waqf). This refers to the recitation rules regarding whether the letter "Ra" (\arb{ر}) in the word "\arb{مصر}" is pronounced with emphasis (\texttt{tafkheem}) or softening (\texttt{tarqeeq}) at the specific pauses in these Surahs. \texttt{wasl}: means not pausing so we only have one way (tafkheem of Raa)
  \end{itemize}

\item \textbf{raa\_nudhur} (\arb{التفخيم والترقيق  في راء \{نذر\} بالقمر وقفا})
  \begin{itemize}
  \item Values: 
    \begin{itemize}
    \item  \texttt{wasl} (\arb{وصل})
    \item  \texttt{tafkheem} (\arb{تفخيم})
    \item  \texttt{tarqeeq} (\arb{ترقيق})
    \end{itemize}
  \item Default Value: \texttt{tafkheem} (\arb{تفخيم})
  \item More Info: Emphasis and softening of the letter 'Ra' in the word \{\arb{نذر}\} in Surah Al-Qamar when pausing (waqf). This refers to the recitation rules regarding whether the letter "Ra" (\arb{ر}) in the word "\arb{نذر}" is pronounced with emphasis (\texttt{tafkheem}) or softening (\texttt{tarqeeq}) when pausing at this word in Surah Al-Qamar. \texttt{wasl}: means not pausing so we only have one way (tarqeeq of Raa)
  \end{itemize}

\item \textbf{raa\_yasr} (\arb{التفخيم والترقيق في راء \{يسر\} بالفجر و\{أن أسر\} بطه والشعراء و\{فأسر\} بهود والحجر والدخان  وقفا})
  \begin{itemize}
  \item Values: 
    \begin{itemize}
    \item  \texttt{wasl} (\arb{وصل})
    \item  \texttt{tafkheem} (\arb{تفخيم})
    \item  \texttt{tarqeeq} (\arb{ترقيق})
    \end{itemize}
  \item Default Value: \texttt{tarqeeq} (\arb{ترقيق})
  \item More Info: Emphasis and softening of the letter 'Ra' in the word \{\arb{يسر}\} in Surah Al-Fajr when pausing (waqf). This refers to the recitation rules regarding whether the letter "Ra" (\arb{ر}) in the word "\arb{يسر}" is pronounced with emphasis (\texttt{tafkheem}) or softening (\texttt{tarqeeq}) when pausing at this word in Surah Al-Fajr. \texttt{wasl}: means not pausing so we only have one way (tarqeeq of Raa)
  \end{itemize}

\item \textbf{meem\_mokhfah} (\arb{هل الميم مخفاة أو مدغمة})
  \begin{itemize}
  \item Values: 
    \begin{itemize}
    \item  \texttt{meem} (\arb{ميم})
    \item  \texttt{ikhfaa} (\arb{إخفاء})
    \end{itemize}
  \item Default Value: \texttt{ikhfaa} (\arb{إخفاء})
  \item More Info: This is not a \textbf{standard} Hafs way but a disagreement between \textbf{scholars} in our century on how to \textbf{pronounce} \textbf{Ikhfa} for meem. Some \textbf{scholars} do full merging (\arb{إدغام}) and the others open the \textbf{lips} a little bit (\arb{إخفاء}). We did not want to add this, but some of the best reciters disagree about this.
  \end{itemize}
\end{itemize}

