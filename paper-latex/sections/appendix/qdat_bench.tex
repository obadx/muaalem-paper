\subsection{QDat Bench Dataset}
\label{sec:qdat_bench}

\subsubsection{Dataset Description}
QDat Bench is a comprehensive benchmark dataset for evaluating model performance in processing Quranic audio recordings with focus on Tajweed rules. This dataset builds upon original qdat dataset \cite{osman2021qdat} after extensive reannotation and enhancement to include all Tajweed rules, complete phoneme-level annotations, and 10 sifat (characteristic) levels.

The enhanced dataset provides F1 and MSE metrics for Tajweed rules, enabling researchers to compare different representations and approaches to Tajweed rule detection. This comprehensive annotation framework makes qdat\_bench particularly valuable for advancing research in automated Quranic pronunciation assessment and error detection.

The dataset addresses several limitations of the original qdat collection:
\begin{itemize}
	\item Incomplete coverage of all Tajweed rules in the original
	\item Multiple reciters recording the same rules (161 reciters with 10 segments each) creating redundancy
	\item Focus on a single representative recitation per rule for more targeted evaluation
	\item Lack of comprehensive phoneme-level and sifat-level annotations
\end{itemize}

QDat Bench contains 159 samples focusing on the verse: \arb{قَالُوا۟ لَا عِلْمَ لَنَآ إِنَّكَ أَنتَ عَلَّٰمُ ٱلْغُيُوبِ} from Surah Al-Ma'idah (5:109), providing a concentrated evaluation of key Tajweed rules.

\subsubsection{Data Structure}
The dataset includes the following main features:
\begin{itemize}
	\item \textbf{audio}: Audio file (sampling rate: None, mono channel).
	\item \textbf{id}: Unique identifier for each element.
	\item \textbf{original\_id}: The item's ID in the original dataset.
	\item \textbf{gender}: Reciter's gender (male/female).
	\item \textbf{age}: Reciter's age.
	\item \textbf{phonetic\_transcript}: Phonetic transcription Quran Phonetic Script~/ref{sec:qps}.
\end{itemize}

\paragraph{Madd (Prolongation) Rules:}
\begin{itemize}
	\item \texttt{qalo\_alif\_len}: Length of normal madd alif in word "قالوا" (0-8)
	\item \texttt{qalo\_waw\_len}: Length of normal madd waw in word "قالوا" (0-8)
	\item \texttt{laa\_alif\_len}: Length of normal madd alif in word "لا" (0-8)
	\item \texttt{separate\_madd}: Length of separate madd for "لنا إنك" (0-8)
	\item \texttt{allam\_alif\_len}: Length of normal madd alif in word "علام" (0-8)
	\item \texttt{madd\_aared\_len}: Length of madd aared for sukoon (0-8)
\end{itemize}

\paragraph{Ghunnah (Nasalization) Rules:}
\begin{itemize}
	\item \texttt{noon\_moshaddadah\_len}: Length of noon moshaddadah in "إنَّك" (0=partial, 1=complete)
	\item \texttt{noon\_mokhfah\_len}: Length of noon mokhfah in "أنت" (0=noon, 1=partial, 2=complete)
\end{itemize}

\paragraph{Qalqalah (Echo) Rules:}
\begin{itemize}
	\item \texttt{qalqalah}: Existence of qalqalah for "الغيوب" (0=no qalqalah, 1=has qalqalah)
\end{itemize}

\paragraph{Letter Characteristics:}
Contains a list of characteristics for each letter in the \texttt{sifat} field:
\begin{itemize}
	\item \texttt{ghonna}: Ghunnah (nasalization)
	\item \texttt{hams\_or\_jahr}: Hams (whisper) or Jahr (clarity)
	\item \texttt{istitala}: Istitala (elongation)
	\item \texttt{itbaq}: Itbaq (adhesion)
	\item \texttt{phonemes}: Phonemes
	\item \texttt{qalqla}: Qalqalah (echo)
	\item \texttt{safeer}: Safeer (whistling)
	\item \texttt{shidda\_or\_rakhawa}: Shidda (strength) or Rakhawa (softness)
	\item \texttt{tafashie}: Tafashie (diffusion)
	\item \texttt{tafkheem\_or\_taqeeq}: Tafkheem (emphasis) or Taqeeq (thinning)
	\item \texttt{tikraar}: Tikraar (repetition)
\end{itemize}

\subsubsection{Dataset Statistics}
\begin{itemize}
	\item Number of samples: 159
	\item Split: train only
	\item Gender distribution: 120 female (75.5\%), 39 male (24.5\%)
	\item Age range: Various age groups represented
\end{itemize}

\begin{figure}[htbp]
	\centering
	\includegraphics[width=0.8\linewidth]{figures/qdat_bench/age_gender_histograms.png}
	\caption{Age and gender distribution of qdat\_bench reciters, showing diverse demographic coverage with 75.5\% female and 24.5\% male participants across different age groups.}
	\label{fig:qdat_age_gender}
\end{figure}

\begin{figure}[htbp]
	\centering
	\includegraphics[width=0.8\linewidth]{figures/qdat_bench/correctness_histogram.png}
	\caption{Distribution of recitation correctness across different Tajweed rules in the benchmark, indicating varying difficulty levels among rule categories and providing insight into common error patterns.}
	\label{fig:qdat_correctness}
\end{figure}

\begin{figure}[htbp]
	\centering
	\includegraphics[width=0.8\linewidth]{figures/qdat_bench/tajweed_columns_histograms.png}
	\caption{Tajweed rules coverage histogram showing the frequency and diversity of rules evaluated in the benchmark dataset, demonstrating comprehensive coverage of key pronunciation aspects.}
	\label{fig:qdat_tajweed_coverage}
\end{figure}

\subsubsection{Evaluation Results}
The detailed evaluation results on qdat\_bench are presented in the following tables, showing performance across different Tajweed rule categories and metrics.

\begin{table}[htbp]
	\centering
	\caption{Detailed QDat Bench Speech Metrics}
	\label{tab:qdat_speech_metrics}
	\begin{tabular}{lc}
		\hline
		\textbf{Metric}           & \textbf{Value} \\
		\hline
		per\_phonemes             & 0.058          \\
		per\_hams\_or\_jahr       & 0.017          \\
		per\_shidda\_or\_rakhawa  & 0.031          \\
		per\_tafkheem\_or\_taqeeq & 0.022          \\
		per\_itbaq                & 0.012          \\
		per\_safeer               & 0.010          \\
		per\_qalqla               & 0.011          \\
		per\_tikraar              & 0.013          \\
		per\_tafashie             & 0.016          \\
		per\_istitala             & 0.009          \\
		per\_ghonna               & 0.014          \\
		average\_per              & 0.019          \\
		\hline
	\end{tabular}
\end{table}

\begin{table}[htbp]
	\centering
	\caption{QDat Bench Madd Rules Performance (RMSE).Golden valuese for madd are: (2 for normal madd, 2 or 4 for spearate madd, and 2, 4, or 6 for aared madd.}
	\label{tab:qdat_madd_metrics}
	\begin{tabular}{lc}
		\hline
		\textbf{Madd Rule}         & \textbf{RMSE} \\
		\hline
		qalo\_alif\_len (normal)            & 0.449         \\
		qalo\_waw\_len (normal)             & 0.456         \\
		laa\_alif\_len (normal)             & 0.404         \\
		separate\_madd (sperate)            & 0.687         \\
		allam\_alif\_len (normal)           & 0.549         \\
		madd\_aared\_len                    & 1.034         \\
		\hline
		\textbf{Average Madd RMSE} & 0.596         \\
		\hline
	\end{tabular}
\end{table}

\begin{table}[htbp]
	\centering
	\caption{QDat Bench Noon Moshaddadah Performance}
	\label{tab:qdat_noon_moshaddadah}
	\begin{tabular}{lccc}
		\hline
		\textbf{Metric} & \textbf{Partial}          & \textbf{Complete} & \textbf{Average} \\
		\hline
		Recall          & 0.659                     & 1.000             & 0.829            \\
		Precision       & 1.000                     & 0.894             & 0.947            \\
		F1 Score        & 0.794                     & 0.944             & 0.869            \\
		Accuracy        & \multicolumn{3}{c}{0.912}                                        \\
		\hline
	\end{tabular}
\end{table}

\begin{table}[htbp]
	\centering
	\caption{QDat Bench Noon Mokhfah Performance}
	\label{tab:qdat_noon_mokhfah}
	\begin{tabular}{lccc}
		\hline
		\textbf{Metric}     & \textbf{Noon}             & \textbf{Partial} & \textbf{Complete} \\
		\hline
		Recall              & 0.468                     & 0.000            & 0.984             \\
		Precision           & 1.000                     & 0.000            & 0.568             \\
		F1 Score            & 0.638                     & 0.000            & 0.720             \\
		\hline
		\textbf{Average F1} & \multicolumn{3}{c}{0.453}                                        \\
		\textbf{Accuracy}   & \multicolumn{3}{c}{0.673}                                        \\
		\hline
	\end{tabular}
\end{table}

\begin{table}[htbp]
	\centering
	\caption{QDat Bench Qalqalah Performance}
	\label{tab:qdat_qalqalah}
	\begin{tabular}{lcc}
		\hline
		\textbf{Metric}   & \textbf{No Qalqalah}      & \textbf{Has Qalqalah} \\
		\hline
		Recall            & 0.966                     & 0.950                 \\
		Precision         & 0.918                     & 0.980                 \\
		F1 Score          & 0.941                     & 0.965                 \\
		\hline
		\textbf{Macro F1} & \multicolumn{2}{c}{0.953}                         \\
		\textbf{Accuracy} & \multicolumn{2}{c}{0.956}                         \\
		\hline
	\end{tabular}
\end{table}

\subsubsection{Performance Analysis and Discussion}

\paragraph{Ikhfaa Detection Challenges:}
The notably low F1 score for Ikhfaa (Noon Mokhfah) at 0.453 reflects several inherent challenges in Quranic pronunciation detection:
\begin{itemize}
	\item \textbf{Acoustic Similarity}: Ikhfaa involves nasalization that is acoustically similar to the clear noon pronunciation (noon moshaddadah), making discrimination challenging for both human listeners and automated systems
	\item \textbf{Limited Ground Truth Reliability}: Few reciters can consistently detect and correctly implement Ikhfaa, leading to potential inconsistencies in the annotation quality even in expert-labeled datasets
	\item \textbf{Context Dependency}: The correct implementation of Ikhfaa depends heavily on the following letter, adding complexity to the detection task
	\item \textbf{Variability in Implementation}: Different reciters may implement Ikhfaa with varying degrees of nasalization, creating inconsistency in the training data
\end{itemize}

\paragraph{Madd Rule Complexity Analysis:}
The Madd (prolongation) rules exhibit varying difficulty levels reflected in their RMSE values:
\begin{itemize}
	\item \textbf{Normal Madd Rules} (RMSE: 0.464): Include basic prolongation rules for alif, waw, and ya with consistent timing patterns, making them relatively easier to model
	\item \textbf{Separate Madd} (RMSE: 0.687): Involves more complex timing requirements due to word boundaries and停顿 considerations
	\item \textbf{Aared Madd} (RMSE: 1.034): The most challenging category, involving prolongation before sukoon with highly variable timing based on recitation style and context
\end{itemize}


\subsubsection{Usage}
The dataset can be loaded using the Hugging Face datasets library:

\begin{verbatim}
from datasets import load_dataset
ds = load_dataset('obadx/qdat_bench')
print(ds['train'][0])  # Display first sample
\end{verbatim}

This benchmark provides a standardized evaluation platform for Quranic pronunciation error detection models, enabling fair comparison across different approaches and facilitating progress in the field of computational Quranic education. The comprehensive nature of the annotations, including all Tajweed rules, complete phoneme coverage, and 10 sifat characteristics, makes qdat\_bench an invaluable resource for researchers developing and comparing different representations of Quranic pronunciation rules.
