\section{Modeling The Quran Phonetic Script}
Our Quran Phonetic script has two outputs: \texttt{phonemes} and \texttt{sifat} (which has 10 attributes). We modeled this as follows: Imagine you are given an input speech utterance and want to output transcripts in Arabic, English, French, and German simultaneously. We implemented this as a speech encoder with a linear layer for each language. Replacing languages with our 11 levels (\texttt{phonemes} and the 10 sifat), we obtain 11 parallel transcription levels. We chose CTC loss \cite{graves2006ctc} without language model integration because we aim to capture what the user actually said, not what they intended to say. We name our architecture \textbf{Multi-level CTC}.  

We compute the loss by averaging all CTC losses for the 11 levels, assigning a weight of 0.4 to the \texttt{phonemes} level as it has the largest vocabulary size (43) compared to other levels.  

\begin{figure}[htbp]
\centering
\includegraphics[width=\linewidth]{figures/multi-level-ctc.png}
\caption{Multi-level CTC loss Architecture composed of 11 Heads for every level and CTC loss for every level with weighted average loss}
\label{fig:multilevel-ctc}
\end{figure}

We fine-tuned Facebook's Wav2Vec2-Bert \cite{barrault2023seamless} for a single epoch with a constant learning rate of \texttt{5e-5} with 64 batch size. We applied augmentations identical to Silero VAD \cite{SileroVAD} using the \texttt{audiomentations} library \cite{Audiomentations}, with additional augmentations: \texttt{TimeStretch} and \texttt{GainTransition}. We filtered out samples longer than \texttt{30 seconds} not due to model limitations, but for efficient GPU utilization - sacrificing only 3k samples out of 250k training samples.  

\begin{figure}[htbp]
\centering
\includegraphics[width=\linewidth]{./figures/audio-lens.png}
\caption{Recitations lengths in seconds for the whole dataset}
\label{fig:audio-lengths}
\end{figure}

The training was done using an H200 GPU with 141 GB of GPU memory for 7 hours.

\paragraph{Resources and Reproducibility}
We release inference code and the Quran phonetization pipeline in the Quran Muaalem and Quran Transcript repositories,\footnote{\url{https://github.com/obadx/quran-muaalem}} \footnote{\url{https://github.com/obadx/quran-transcript}} and provide the datasets and benchmarks on Hugging Face.\footnote{\url{https://huggingface.co/datasets/obadx/muaalem-annotated-v3}} \footnote{\url{https://huggingface.co/datasets/obadx/qdat_bench}} Supporting data‑preparation tools (collection and segmentation) are available in companion repositories.\footnote{\url{https://github.com/obadx/prepare-quran-dataset}} \footnote{\url{https://github.com/obadx/recitations-segmenter}}
