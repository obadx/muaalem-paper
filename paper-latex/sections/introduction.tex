\section{Introduction}
Assessing pronunciation is not a simple task \cite{kheir2023automatic}, as it does not only rely on pronouncing phonemes correctly but also involves other factors like intonation, prosody, and stress. Does learning these mean one is done? No---other factors include fluency and completeness \cite{kheir2023automatic}. However, the Holy Quran presents unique characteristics: it is among the easiest spoken texts to learn despite containing special phonemes absent in other languages.

The pronunciation of the Holy Quran is governed by rigorously strict rules formally defined by ancient Muslim scholars since the 6th century. Despite their beauty and precision, these rules have not been comprehensively digitized (to our knowledge) for Quranic pronunciation assessment.

Although RDI pioneered computer-aided Quranic instruction \cite{sherif2007enhancing}, they neither disclosed their phoneticization process nor released data/models. Consequently, new research must start from basics: defining phoneticization, data, and models. To bridge this gap, we introduce:

\begin{itemize}
	\item \textbf{A Phonetizer}: Encodes \emph{all} Tajweed rules and articulation attributes (\emph{Sifat}) defined by classical scholars, except \emph{Ishmam} (\arb{إشمام})
	\item \textbf{A 98\% automated pipeline}: Generates highly accurate datasets from expert recitations
	\item \textbf{A dataset}: $\sim$300K annotated utterances (850+ hours)
	\item \textbf{qdat\_bench}: Benchmarks phonemes, diacritization, and Tajweed rules (Ghunnah, Qalqalah, Madd) on real recitation errors containing 159 samples
	\item \textbf{Integration}: Our multi-level CTC model proves the Quranic phonetic script is learnable (0.21\% average phoneme error rate)
\end{itemize}

The paper is organized as follows:
\begin{itemize}
	\item \textbf{Related Work}: Expands on strengths/weaknesses of prior research
	\item \textbf{Quran Phonetic Script}: Introduces our two-level script: \textbf{phonemes} and \textbf{Sifat} (10 attributes $\to$ 11 total levels)
	\item \textbf{Data Pipeline}: Stages include:
	      \begin{enumerate}
		      \item Digitized Quran script as foundation
		      \item \emph{Hafs} methodology criteria
		      \item Expert recitation collection
		      \item Segmentation at pause points (\arb{وقف})
		      \item Segment transcription
		      \item Validation via \emph{Tasmee} (\arb{تسميع}) algorithm
	      \end{enumerate}
	\item \textbf{Modeling}: Demonstrates learnability of the phonetic script
	\item \textbf{Results}: Analysis of outcomes
	\item \textbf{Limitations \& Future Work}: Next research directions
	\item \textbf{Conclusion}: Summary of contributions
	\item \textbf{Appendix}: Details on \emph{Mushaf} attributes and algorithms
\end{itemize}

